%!TEX root = ../../main.tex
\chapter{Experimente}
Die Bewertung eines \glspl{Modell} ist ein wichtiger Schritt in der Entwicklung, um zu beurteilen ob das \gls{Modell} gute Leistungen erzielt. Es gibt verschieden Methoden bzw. Berechnungen, um die Leistungsfähigkeit zu untersuchen. Es werden nachfolgend verschiedene Metriken, sowie die Ergebnisse der \gls{Modell}e vorgestellt.
\section{Metriken}
Metriken sind bestimmte Funktionen, anhand derer die Leistung eines \glspl{Modell} bewertet werden kann. Sie berechnen die Übereinstimmung zwischen der Vorhersage und dem tatsächlichen Ergebnis. Abhängig von der Problemstellung sind eine oder mehrere geeignete Metriken auszuwählen. Eine einzelne Metrik reicht oft nicht aus für eine allumfassende Bewertung, deshalb werden meist mehrere Metriken betrachtet.
\subsection{IoU}
\subsection{DICE}
\subsection{Accuracy}
\section{Diskussion der Ergebnisse}
\section{Fehleranalyse und Verbesserungen}
