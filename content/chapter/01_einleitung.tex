%!TEX root = ../../main.tex

\chapter{Einleitung}
\section{Einführung und Hintergrund}
In der Medizin gibt es schon seit mehr als 50 Jahren bildgebende Verfahren, mit denen Aufnahmen vom Inneren des menschlichen Körpers möglich sind. Diese Verfahren wurden kontinuierlich weiterentwickelt und sind inzwischen in der Lage detaillierte Bilder zu erzeugen. Technologien wie eine Computertomographie (CT) oder eine \ac{MRT} sind in der Lage präzise Informationen über innere Organe, Strukturen und Gewebe im Körper zu liefern. \cite[vgl.][]{Heinrichs2022} Mit den modernen bildgebenden Verfahren, steigt die Datenmenge und Komplexität, wie auch die Herausforderung der Analyse und Verarbeitung der Daten bzw. Bilder. 

Aufgrund der immer weiter fortschreitenden Technologien, setzen immer mehr medizinische Institutionen auf bildgebende Verfahren, um so Krankheiten zu diagnostizieren und behandeln. Allerdings ist die Analyse und Verarbeitung dieser Daten mit der Zeit komplexer geworden, sodass Ärzte und Mediziner vor eine Herausforderung gestellt werden. Die manuelle Analyse und Verarbeitung der Daten bringt oft Schwierigkeiten und Fehlerquellen mit sich. Der Prozess ist meist sehr zeitintensiv, weshalb es häufig zu langen Wartezeiten bei Auswertungen für die Patienten kommen kann. Nicht nur die Wartezeit ist ein Problem, sondern auch menschliche Fehler bei Interpretation der Daten, welche zu Fehldiagnosen oder -behandlungen führen können. \cite[vgl.][]{Ramsundar2020}

Diese Herausforderungen können mithilfe des vielversprechenden Ansatzes der Automatisierung gelöst werden. So kann bspw. die Effizienz gesteigert und die Fehlerrate gesenkt werden. Ein wichtiges Stichwort ist hierbei \ac{KI}, die ebenfalls in den vergangenen Jahren enorme Fortschritte gemacht hat. Insbesondere im Bereich der Bildanalyse haben \ac{KI}-Systeme durch die Entwicklung von Deep Learning-\gls{Modell}en enorm an Genauigkeit gewonnen. Die Anwendung von Deep Learning auf medizinische Bilddaten bietet vielversprechende Möglichkeiten für eine verbesserte Diagnose, Therapieplanung und -überwachung sowie für die Entwicklung von personalisierten Medizinlösungen. 

Besonders im medizinischen Bereich hat die Nutzung von \ac{KI}-Systemen auf medizinischen Bilddaten in den letzten Jahren an Bedeutung gewonnen. Mithilfe von Deep Learning-Modellen können Ärzte und Forscher komplexe Bilder von medizinischen Bildgebungsverfahren wie Röntgenaufnahmen, CT-Scans und MRT-Scans automatisiert analysieren. Diese Analyse kann zur Erkennung von Krankheiten, Identifizierung von Tumoren, sowie zur Bewertung der Therapieeffektivität eingesetzt werden. 


\section{Problemstellung und Zielsetzung}
Die zunehmende Komplexität und Menge an medizinischen Daten und Bildern stellt Mediziner vor große Herausforderungen. Die manuelle Analyse und Verarbeitung der Daten ist zeitaufwendig, fehleranfällig und es besteht die Gefahr wichtige Sachen zu übersehen, welche für die Diagnose nützlich sind.
Gerade im Bereich der Tumore, also der Krebszellen, ist die frühzeitige Erkennung aber auch die Behandlung sehr wichtig für das Wohl des Patienten. 

Vor allem im Bereich des Gehirns ist mit Tumoren sehr behutsam umzugehen, um so keine bleibenden Schäden beim Patienten zu hinterlassen. Bei Gehirntumoren ist es wichtig die genaue Position und Lage zu kennen, damit eine geeignete Behandlung gefunden werden kann. Die Operation bei einem Gehirntumor bedarf großer Vorbereitung und ist komplex, da Fachärzte zunächst den Tumor genau segmentieren müssen, bevor sie weitere Schritte planen. Diese Vorbereitung ist unter anderem zeitintensiv und kann nur von ausgebildeten Fachärzten übernommen werden. 

Befindet sich ein Tumor im Gehirn, so bedarf es einer noch genaueren Vorbereitung und Arbeit, um diesen zu entfernen. 
Krebs ist ein vielseitiges Thema, mit dem sich die Medizin seit Jahren beschäftigt. Heutzutage setzen Ärzte vor allem bei der frühzeitigen Erkennung von Tumoren auf die Hilfe von \ac{KI}. Doch nicht nur bei der Früherkennung, sondern auch bei der Behandlung solcher Tumore. Sind solche im Bereich des Gehirns, ist es wichtig behutsam vorzugehen, um so keine bleibenden Schäden zu verursachen.

Die Problemstellung dieser Studienarbeit ist die Identifizierung und Klassifizierung von Tumoren im Gehirn, genauer gesagt eine Multi Klassen Segmentierung, bei welcher jeder Bildpunkt eine Tumor Klasse zugewiesen bekommt. Die verwendeten Daten sind aus dem öffentlichen Datensatz BraTS (Brain Tumor Segmentation Challenge), welcher aus rund 1250 \ac{MRT}-Bildern von Gehirnen besteht, welche von Fachärzten auf Tumore untersucht und wenn vorhanden markiert wurden. Eine Herausforderung dabei ist es, ein geeignetes Deep Learning \gls{Modell} zu entwickeln, dass für die Segmentierung von Tumoren im Gehirn geeignet ist und möglichst genaue Ergebnisse liefert.  

Die Zielsetzung dieser Arbeit ist die Erstellung eines \gls{Modell}s für die Segmentierung von Gehirntumoren, anhand des BraTS Datensatzes. Das \gls{Modell} soll in der Lage sein zwischen drei verschiedenen Tumor-Klassen zu unterscheiden und für diese eine Maske zu erstellen. Für die Entwicklung des Modells wird das Open Source Machine-Learning Framework PyTorch verwendet, welches von Facebook entwickelt wurde. Das Framework bietet zahlreiche Tools und Funktionen für das Entwickeln von Deep Learning-\glspl{Modell}. \cite[vgl.][]{PyTorch}