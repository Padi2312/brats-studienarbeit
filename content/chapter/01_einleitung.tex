%!TEX root = ../../main.tex

\chapter{Einleitung}
\section{Einführung und Hintergrund}
Seit mehr als 50 Jahren gibt es in der Medizin bildgebende Verfahren, mit denen Aufnahmen aus dem Inneren des menschlichen Körpers gemacht werden können. Diese Verfahren wurden kontinuierlich weiterentwickelt und sind heute in der Lage, detaillierte Bilder zu erzeugen. Technologien wie die Computertomographie (CT) oder die Magnetresonanztomographie (\ac{MRT}) sind in der Lage, präzise Informationen über innere Organe, Strukturen und Gewebe im Körper zu liefern. \cite[vgl.][]{Heinrichs2022} Mit den modernen bildgebenden Verfahren steigt die Datenmenge und -komplexität und damit auch die Herausforderung an die Analyse und Verarbeitung der Daten bzw. Bilder.

Aufgrund der sich ständig weiterentwickelnden Technologien setzen immer mehr medizinische Einrichtungen bildgebende Verfahren zur Diagnose und Behandlung von Krankheiten ein. Die Analyse und Verarbeitung dieser Daten ist jedoch im Laufe der Zeit immer komplexer geworden und stellt eine Herausforderung für Ärzte und medizinisches Personal dar. Die manuelle Analyse und Verarbeitung der Daten ist oft mit Schwierigkeiten und Fehlerquellen verbunden. Der Prozess ist in der Regel sehr zeitaufwendig, weshalb es häufig zu langen Wartezeiten bei der Auswertung für die Patienten kommen kann. Nicht nur die Wartezeit ist ein Problem, sondern auch menschliche Fehler bei der Interpretation der Daten, die zu Fehldiagnosen oder Fehlbehandlungen führen können. \cite[vgl.][]{Ramsundar2020}

Diesen Herausforderungen kann mit dem vielversprechenden Ansatz der Automatisierung begegnet werden. So kann beispielsweise die Effizienz gesteigert und die Fehlerquote gesenkt werden. Ein wichtiges Stichwort ist hier die \ac{KI}, die in den letzten Jahren ebenfalls enorme Fortschritte gemacht hat. Insbesondere im Bereich der Bildanalyse haben \ac{KI}-Systeme durch die Entwicklung von Deep-Learning-\gls{Modell}en enorm an Genauigkeit gewonnen. Die Anwendung von Deep Learning auf medizinische Bilddaten bietet vielversprechende Möglichkeiten für eine verbesserte Diagnose, Therapieplanung und -überwachung sowie für die Entwicklung personalisierter medizinischer Lösungen.

Gerade im medizinischen Bereich hat der Einsatz von \ac{KI}-Systemen auf medizinischen Bilddaten in den letzten Jahren an Bedeutung gewonnen. Mit Hilfe von Deep-Learning-Modellen können Ärzte und Forscher komplexe Bilder aus bildgebenden Verfahren wie Röntgenaufnahmen, CT- und MRT-Scans automatisiert analysieren. Diese Analyse kann zur Erkennung von Krankheiten, zur Identifizierung von Tumoren und zur Bewertung der Wirksamkeit von Therapien eingesetzt werden. 


\section{Problemstellung und Zielsetzung}
Die zunehmende Komplexität und Menge medizinischer Daten und Bilder stellt Ärzte vor große Herausforderungen. Die manuelle Analyse und Verarbeitung der Daten ist zeitaufwändig, fehleranfällig und birgt die Gefahr, dass wichtige diagnostische Informationen übersehen werden.
Gerade im Bereich von Tumoren, also Krebszellen, ist die Früherkennung, aber auch die Behandlung sehr wichtig für das Wohl des Patienten. 

Vor allem im Bereich des Gehirns muss mit Tumoren sehr behutsam umgegangen werden, um keine bleibenden Schäden beim Patienten zu hinterlassen. Bei Hirntumoren ist es wichtig, die genaue Position und Lage zu kennen, um eine geeignete Behandlung zu finden. Die Operation eines Hirntumors erfordert eine umfangreiche Vorbereitung und ist komplex, da die Fachärzte den Tumor zunächst genau segmentieren müssen, bevor sie die weiteren Schritte planen. Diese Vorbereitung ist zeitaufwändig und kann nur von ausgebildeten Fachärzten durchgeführt werden. 

Krebs ist ein vielschichtiges Thema, mit dem sich die Medizin seit vielen Jahren beschäftigt. Vor allem bei der Früherkennung von Tumoren setzen Ärzte heute auf die Hilfe der \ac{KI}. Aber nicht nur bei der Früherkennung, sondern auch bei der Behandlung solcher Tumore.

Die Problemstellung dieser Studienarbeit ist die Identifizierung und Klassifizierung von Tumoren im Gehirn, genauer gesagt eine Multiklassen-Segmentierung, bei der jedem Bildpunkt eine Tumorklasse zugeordnet wird. Die verwendeten Daten stammen aus dem öffentlichen Datensatz BraTS (Brain Tumor Segmentation Challenge), der aus ca. 1250 \ac{MRT}-Bildern von Gehirnen besteht, die von Fachärzten auf Tumore untersucht und gegebenenfalls markiert wurden. Die Herausforderung besteht darin, ein geeignetes Deep-Learning-Modell zu entwickeln, das für die Segmentierung von Tumoren im Gehirn geeignet ist und möglichst genaue Ergebnisse liefert.  

Das Ziel dieser Arbeit ist es, ein \gls{Modell} für die Segmentierung von Gehirntumoren auf Basis des BraTS Datensatzes zu erstellen. Das \gls{Modell} soll in der Lage sein, zwischen drei verschiedenen Tumorklassen zu unterscheiden und für diese eine Maske zu erstellen. Für die Entwicklung des Modells wird das von Facebook entwickelte Open Source Machine-Learning Framework PyTorch verwendet. Das Framework bietet zahlreiche Werkzeuge und Funktionen für die Entwicklung von Deep Learning-\glspl{Modell}. \cite[vgl.][]{PyTorch}