%!TEX root = ../../main.tex
\chapter{Zusammenfassung und Ausblick}
% Überarbeiten
Die vorliegende Arbeit hat sich der wichtigen Aufgabe angenommen, ein Convolutional Neural Network (CNN), speziell das U-Net, für die Segmentierung von Hirntumoren in MRT-Bilddaten zu trainieren und zu evaluieren. Im Rahmen dieser Arbeit wurde ein detailliertes Verständnis der Architektur und Funktionsweise des U-Nets sowie der damit verbundenen Konzepte wie Backpropagation, Aktivierungsfunktionen, Verlustfunktionen und Hyperparameter erarbeitet. Die Vorverarbeitung des BraTS 2021-Datensatzes wurde gründlich durchgeführt, um die Daten für das Training des neuronalen Netzwerks optimal vorzubereiten.

Die erzielten Resultate zeigten vielversprechende Ergebnisse hinsichtlich der Fähigkeit des trainierten U-Nets, Hirntumoren in MRT-Bilddaten zu segmentieren. Die Diskussion und Fehleranalyse zeigte jedoch auch Bereiche auf, in denen Verbesserungen möglich und notwendig sind, um die Leistungsfähigkeit des Modells weiter zu erhöhen. Insbesondere wurden die Auswirkungen verschiedener Hyperparameter und Vorverarbeitungsmethoden auf die Modellleistung erörtert.

Das Feld der medizinischen Bildsegmentierung, und speziell die Segmentierung von Hirntumoren in MRT-Bilddaten, ist von großer Bedeutung, da es direkte Auswirkungen auf die Diagnose und Behandlung von Patienten hat. Die in dieser Arbeit vorgestellte Methode bietet einen wertvollen Beitrag zu diesem Forschungsfeld, und die erzielten Ergebnisse legen nahe, dass die weitere Verbesserung und Anpassung des Ansatzes zu noch besseren Ergebnissen führen könnte.

In Bezug auf zukünftige Forschungsrichtungen und Verbesserungen könnten verschiedene Aspekte in Betracht gezogen werden. Einerseits könnten alternative Netzwerkarchitekturen untersucht und evaluiert werden, um zu sehen, ob sie möglicherweise bessere Ergebnisse liefern. Andererseits könnte auch eine detailliertere Untersuchung der Auswirkungen von Vorverarbeitungsverfahren und Hyperparametern durchgeführt werden, um den Trainingsprozess und die Modellleistung weiter zu optimieren. Darüber hinaus könnte die Integration von zusätzlichen Datenquellen oder Modalitäten in den Trainingsprozess eine wertvolle Erweiterung des Ansatzes darstellen und zur Verbesserung der Ergebnisse beitragen.

Abschließend kann festgehalten werden, dass trotz der erzielten Fortschritte und Erfolge in der vorliegenden Arbeit noch viele spannende und wichtige Forschungsrichtungen in der Segmentierung von Hirntumoren in MRT-Bilddaten offen bleiben, die in zukünftigen Arbeiten angegangen werden sollten.