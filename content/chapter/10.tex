%!TEX root = ../../main.tex
\chapter{Zusammenfassung und Ausblick}
Die vorliegende Studienarbeit hat sich mit dem Einsatz von Deep Learning, insbesondere mit der U-Net-Architektur, für die Segmentierung von Hirntumoren in MRT-Bildern beschäftigt. Durch den Einsatz künstlicher neuronaler Netze konnte eine Methode zur automatisierten Segmentierung erarbeitet werden.

Im Rahmen der Arbeit wurden zunächst die theoretischen Grundlagen des Deep Learning und speziell von \ac{CNN} und der U-Net-Architektur erläutert. Die U-Net-Architektur hat sich dabei als geeignet für die Segmentierungsaufgaben in der medizinischen Bildgebung erwiesen. Sie verbindet die Vorteile von Feature-Erkennung und -Lokalisierung, was zu guten Segmentierungsergebnissen führt.

Der verwendete Datensatz, \ac{BraTS}, bietet eine Vielzahl an \ac{MRT}-Bildern mit unterschiedlichen Hirntumoren, welche für das Training eines \gls{Modell}s verwendet werden können. Bei der Datenvorverarbeitung wurde auf Aspekte wie Skalierung und Aufteilung eingegangen, um eine optimale Vorbereitung für das Training des neuronalen Netzes zu gewährleisten. Im Zuge der Modellentwicklung wurden verschiedene Hyperparameter evaluiert und eine effiziente Implementierung des Modells vorgestellt.

Die durchgeführten Experimente zeigen die Auswirkungen der verschiedenen Hyperparameter und welche Parameterkombinationen für das Training des Modells am besten geeignet sind. Die Evaluierung der Modelle erfolgte anhand verschiedener Metriken und lieferte wertvolle Ergebnisse. Ebenso zeigte die Fehleranalyse, dass es noch Verbesserungs- und Optimierungsmöglichkeiten gibt.

Die Aussichten für weitere Verbesserungen in diesem Projekt sind vielversprechend. Die Performance des Modells könnte durch die Verwendung anderer Modellarchitekturen oder eines größeren Datensatzes optimiert werden. Außerdem könnten zusätzliche Datensätze, z.B. auch ältere \ac{BraTS}-Datensätze, in das Training einbezogen werden, um die Generalisierbarkeit des Modells weiter zu verbessern. 

Abschließend kann festgehalten werden, dass trotz der in dieser Arbeit erzielten Ergebnisse noch viele weitere Themen und Verbesserungen im Hinblick auf die Segmentierung von Hirntumoren bearbeitet werden können, um bessere Ergebnisse zu erzielen.