%!TEX root = ../../main.tex
\chapter{Zusammenfassung und Ausblick}
Die vorliegende Studienarbeit hat sich mit dem Einsatz von Deep Learning, insbesondere mit der U-Net-Architektur, für die Segmentierung von Hirntumoren in MRT-Bildern beschäftigt. Durch den Einsatz künstlicher neuronaler Netze konnte eine effektive Methode zur automatisierten Segmentierung vorgestellt werden, die das Potenzial hat, die Diagnostik und Behandlungsplanung in der Neuroonkologie zu verbessern.

Im Rahmen der Arbeit wurden zunächst die theoretischen Grundlagen des Deep Learning und speziell von \ac{CNN} und der U-Net-Architektur erläutert. Die U-Net-Architektur hat sich dabei als geeignet für die Segmentierungsaufgaben in der medizinischen Bildgebung erwiesen. Sie verbindet die Vorteile von Feature-Erkennung und -Lokalisierung, was zu guten Segmentierungsergebnissen führt.

Der verwendete Datensatz, \ac{BraTS}, bietet eine Vielzahl an \ac{MRT}-Bildern mit unterschiedlichen Hirntumoren. Bei der Datenvorverarbeitung wurde auf Aspekte wie Skalierung und Aufteilung eingegangen, um eine optimale Vorbereitung für das Training des neuronalen Netzes zu gewährleisten. Im Zuge der Modellentwicklung wurden verschiedene Hyperparameter evaluiert und eine effiziente Implementierung des Modells vorgestellt.

Die durchgeführten Experimente demonstrieren die Leistungsfähigkeit des Modells. Während die Bewertung der Modelle auf Basis verschiedener Metriken erfolgte, hat die Fehleranalyse wertvolle Einblicke in mögliche Verbesserungen und Optimierungen gegeben.

Der Ausblick auf zukünftige Projekte und Verbesserungen in diesem Bereich ist vielversprechend. Zum einen könnte die Leistung des Modells durch den Einsatz von anderen bzw. moderneren Modellarchitekturen oder verbesserten Trainingstechniken weiter optimiert werden. Zum anderen könnten zusätzliche Datensätze in das Training einbezogen werden, um die Generalisierbarkeit des Modells weiter zu verbessern. Darüber hinaus wäre es interessant, den Einsatz von semi- oder unüberwachtem Lernen zur weiteren Verbesserung der Segmentierungsleistung zu erforschen.

Abschließend kann festgehalten werden, dass trotz der erzielten Ergebnisse und Erfolge in der vorliegenden Arbeit, noch viele weitere Themen und Verbesserungen behandelt werden sollten, im Hinblick auf die Segmentierung von Hirntumoren, um bessere Ergebnisse zu erzielen.