%!TEX root = ../main.tex

\pagestyle{empty}

% override abstract headline
\renewcommand{\abstractname}{Abstract}

\begin{abstract}
Diese Studienarbeit bietet eine Einführung in das Thema Deep Learning, speziell im medizinischen Bereich. In dieser Arbeit wurde ein Modell entwickelt, welches Gehirntumore auf MRT Bildern segmentiert. Ein besonderer Fokus lag auf dem Einsatz der U-Net Architektur, einer speziellen Form eines Convolutional Neural Networks (CNNs), die sich durch ihre effiziente und genaue Segmentierungsleistung in biomedizinischen Anwendungen auszeichnet.

Die Arbeit umfasste die Vorverarbeitung des Brain Tumor Segmentation Challenge (BraTS) 21 Datensatzes, einschließlich Normalisierung, Größenskalierung und Daten-Augmentation, um die Qualität und Vielfalt der Trainingsdaten zu erhöhen. Darüber hinaus wurden Konzepte wie Backpropagation, Aktivierungsfunktionen, Verlustfunktionen und das Einstellen von Hyperparametern behandelt, um den Trainingprozess des U-Nets zu optimieren.

Die erzielten Ergebnisse zeigten, dass das trainierte U-Net in der Lage ist, Hirntumoren in MRT-Bildern mit einer guten Leistung zu segmentieren. Dennoch wurden auch einige Bereiche identifiziert, in denen Verbesserungen möglich sind, wie z.B. die Feinabstimmung von Hyperparametern und die Erweiterung von Vorverarbeitungstechniken. Die Ergebnisse der Arbeit wurden für die Einführung erarbeitet und bieten einen Einblick was alles möglich ist mit Deep Learning und Künstlicher Intelligenz.
\end{abstract}