%!TEX root = ../main.tex

\pagestyle{empty}

% override abstract headline
\renewcommand{\abstractname}{Abstract}

\begin{abstract}
In dieser Studienarbeit wurden Methoden zur automatisierten Segmentierung von Hirntumoren in MRT-Bildern mittels Deep Learning untersucht. Ein besonderer Fokus lag auf dem Einsatz des U-Net-Modells, einer spezialisierten Form von Convolutional Neural Networks (CNNs), die sich durch ihre effiziente und genaue Segmentierungsleistung in biomedizinischen Anwendungen auszeichnet.

Die Arbeit umfasste die ausführliche Vorverarbeitung des Brain Tumor Segmentation Challenge (BraTS) 2021-Datensatzes, einschließlich Normalisierung, Größenskalierung und Daten-Augmentation, um die Qualität und Vielfalt der Trainingsdaten zu erhöhen. Darüber hinaus wurden Konzepte wie Backpropagation, Aktivierungsfunktionen, Verlustfunktionen und das Einstellen von Hyperparametern behandelt, um den Trainingprozess des U-Nets zu optimieren.

Die erzielten Ergebnisse zeigten, dass das trainierte U-Net in der Lage ist, Hirntumoren in MRT-Bildern mit einer guten Leistung zu segmentieren. Dennoch wurden auch einige Bereiche identifiziert, in denen Verbesserungen möglich sind, wie z.B. die Feinabstimmung von Hyperparametern und die Erweiterung von Vorverarbeitungstechniken. Die Ergebnisse dieser Arbeit tragen zu dem aufstrebenden Forschungsfeld der automatisierten Hirntumorsegmentierung bei und bieten wichtige Einblicke für zukünftige Arbeiten in diesem Bereich.
\end{abstract}