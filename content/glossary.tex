%!TEX root = ../main.tex

%
% To create glossary run the following command: 
% makeglossaries main.acn && makeglossaries main.glo
%

%
% Glossareintraege --> referenz, name, beschreibung
% Aufruf mit \gls{...}
%
\newglossaryentry{Glossareintrag}{name={Glossareintrag},plural={Glossareinträge},description={Ein Glossar beschreibt verschiedenste Dinge in kurzen Worten}}
\newglossaryentry{Algorithmus}{name={Algorithmus},plural={Algorithmen},description={Ein Algorithmus ist eine genau definierte Folge von Aktionen, für die Lösung eines Problems\cite[vgl.][]{Algorithmus}}}
\newglossaryentry{Modell}{name={Modell},plural={Modelle},description={Ein Modell beschreibt eine komplexe mathematische Funktion, das anhand von Daten trainiert wurde, um bestimmte Vorhersagen oder Entscheidungen zu treffen. \cite[vgl.][]{MSModell}}}
\newglossaryentry{Ionisierende Strahlung}{name={Ionisierende Strahlung},description={Strahlung die so stark ist, das sie Elektronen aus Atomen oder Molekühlen entfernt kann, sodass diese positiv geladen sind (ionisiert). Diese Art von Strahlung kann Mensch und Umwelt schädigen.\cite[vgl.][]{IonisierendeStrahlung}}}
\newglossaryentry{Proton}{name={Proton},plural={Protonen},description={Ein stabiles, elektrisch positiv geladenes Teilchen.}}
